\documentclass{scrartcl}

\title{An XPath Query Evaluator for Filesystems}
\subtitle{EECS 433 Final Project}
\date{December 8, 2016}
\author{Stephen Brennan}

\usepackage[utf8]{inputenc}
\usepackage[letterpaper, margin=1in]{geometry}
\usepackage{listings}
\usepackage{courier}
\usepackage{color}
\usepackage{hyperref}

\definecolor{mygreen}{rgb}{0,0.6,0}
\definecolor{mygray}{rgb}{0.5,0.5,0.5}
\definecolor{mymauve}{rgb}{0.58,0,0.82}
\lstset{%
  backgroundcolor=\color{white},     % background color
  basicstyle=\footnotesize\ttfamily, % monospace, small font
  breaklines=true,                   % nice line breaks
  captionpos=b,                      % put captions at bottom
  commentstyle=\color{mygreen},      % comments
  escapeinside={\%*}{*)},            % if you want to add LaTeX in code
  keywordstyle=\color{blue},         % keyword
  stringstyle=\color{mymauve},       % string
}

\begin{document}
\maketitle

\abstract

XPath is a query language designed for addressing the nodes of an XML file.
However, its power in addressing and querying tree-structured data makes it
applicable to other types of data. In this project, we describe our
implementation of one such application: DPath. DPath is a tool that allows users
to search for files with XPath expressions.

\section{Introduction}

eXtensible Markup Language (XML) is a standardized data format for sharing and
storing structured data. It bears a close resemblance to Hypertext Markup
Language (HTML), the fundamental markup underlying the web. This is because both
languages are based on Structured Generalized Markup Language, a
``meta-language'' for defining documents that contain ``markup'' \cite{sgml}.
``Markup'' can be thought of as a way to annotate data, creating structure. HTML
uses markup to describe how content should be formated. XML uses markup to
define logical structure of data. The fundamental unit of structure in XML is a
tag:

\begin{lstlisting}[language=XML]
  <Student></Student>
\end{lstlisting}

Tags may have textual attributes associated with them:

\begin{lstlisting}[language=XML]
  <Student id="1234" name="Stephen Brennan"></Student>
\end{lstlisting}

Tags can contain textual data as well as other tags:

\begin{lstlisting}[language=XML]
  <Student id="1234" name="Stephen Brennan">
    <Course id="EECS 433">Database Systems</Course>
    <Course id="MATH 303">Number Theory</Course>
    <Course id="EECS 651">Master&apos;s Thesis</Course>
  </Student>
\end{lstlisting}

XML tags, attributes, and data form a tree structure. XPath is a query language
that can be used to query and address each part of tree structures \cite{xpath}.
It achieves this using \emph{path expressions}. For example, the path expression
\texttt{//Student/Course} describes all \texttt{Course} nodes whose parent is a
\texttt{Student} node. XPath allows for even more fine-grained queries through
the use of predicates, as well as several different \emph{axes}. All of these
concepts will be discussed at length in Section~\ref{sec:xpath}.

Since XPath is a language for querying the XML tree structure, it could also be
applicable to other tree structures. One common tree structure encountered in
everyday computing is the file system. Most modern file systems are organized
into a tree where internal nodes are directories and leaf nodes are files (or
empty directories). Like XML, each file and directory has a name and attributes.
However, there are some important differences between file systems that should
be recognized.

\begin{itemize}
\item Unlike XML, a fully qualified path file path uniquely identifies a file,
  whereas a fully qualified tag path in XML may identify a whole set of nodes.
  This is because there may be several sibling tags with the same name in XML.
\item File systems may contain \emph{links} which point to other parts of the
  system, or even to a parent, creating cycles. For the purpose of this paper,
  we will ignore these links.
\end{itemize}

For this project, we have implemented a tool which evaluates queries in a
language nearly identical to XPath, called DPath. This tool runs from the
command line, accepts a query as its sole argument, and outputs one result per
line on standard output. An example query is presented below.

\begin{lstlisting}
stephen at greed in ~/go/src/github.com/brenns10/dpath
$ dpath './.[starts-with(name(), ../name())]'
file:/home/stephen/go/src/github.com/brenns10/dpath/dpath.nn.go
file:/home/stephen/go/src/github.com/brenns10/dpath/dpath.y
file:/home/stephen/go/src/github.com/brenns10/dpath/dpath
file:/home/stephen/go/src/github.com/brenns10/dpath/dpath.nex
\end{lstlisting}

This query returns files from the DPath source directory which start with the
same name as their containing directory.

The remainder of this report will be organized as follows.
Section~\ref{sec:xpath} will give a detailed description of the XPath query
language's syntax and semantics. Section~\ref{sec:dpath} will describe the
implementation of DPath. Section~\ref{sec:future} will discuss future work for
this project, and Section~\ref{sec:related} will discuss related work.

\section{XPath}
\label{sec:xpath}

The XPath language is standardized by the World Wide Web Consortium (W3C), and
it is used within the XQuery and XSLT languages, which are also W3C standards
\cite{xpath}. In the research and development of DPath, we focused on the 2.0
version of the XPath standard.

There are several parts to the XPath standard: syntax\cite{xpath},
semantics\cite{xpath-semantics}, data model\cite{xpath-datamodel}, type system,
operators, and built-in function library\cite{xpath-functions}.

\section{DPath Implementation}
\label{sec:dpath}

\section{Future Work}
\label{sec:future}

\section{Related Work}
\label{sec:related}


\bibliographystyle{ieeetr}
\bibliography{report}
\end{document}
